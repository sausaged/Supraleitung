\documentclass[a4paper]{scrartcl}

\usepackage[utf8]{inputenc}
\usepackage[ngerman]{babel}
\usepackage[T1]{fontenc}
\usepackage{hyperref}
\usepackage{graphicx}
\usepackage{tikz}
\usetikzlibrary{arrows.meta}


\usepackage{amsfonts}
\usepackage{amsmath}

\title{Supraleitung}
\author{Jonas Leggewie}
\date{\today}

\begin{document}
\maketitle
\newpage
\tableofcontents
\newpage


\section{Einleitung}
Supraleitung beschreibt das Phänomen, bei dem elektrischer Strom, bei sehr niedrigen Temperaturen 
ohne Wiederstand fließt, wodurch eine verlustfreie Energieübertragung möglich Wird. Was die 
Übertragungsverluste auf langen Stromtrassen deutlich reduzieren würde und eine Effizientere 
Energieversorgung ermöglichen könnte. Auserdem haben Supraleiter noch andere spannende 
Eigenschaften, welche noch nicht vollständig verstanden wurden. 


\section{Stromleitung auf der Atomarer Ebene}
Strom fließt in einen Leiter, z.B ein Kupferdraht, indem sich Elektronen durch das Leitermaterial
bewegen. Elektrische Leiter bestehen aus positive geladenen Ionenrümpfe\footnote{Ionenrümpfe sind die Atome des Kristallgitters, die eines oder mehrere ihrer außen Elektronen abgegeben haben und deswegen positiv geladen sind}, die in einem Kristallgitter angeordnet sind. Um 
die Ionenrümpfe befinden sich die Elektronen welche wie in einem Gas frei um die Ionenrümpfe herum fliegen.


\begin{figure}[h]
    
    \centering
    \begin{tikzpicture}
        % Zeichne die Ionenrümpfe als Kreise im Kristallgitter
        \foreach \x in {0,2,4,6} {
            \foreach \y in {0,2,4} {
                \node[circle, draw=gray!80, fill=gray!40, minimum size=7.5mm] at (\x,\y) {};
                \draw[line width=0.5mm] (\x-0.1,\y) -- (\x+0.1,\y);
                \draw[line width=0.5mm] (\x,\y-0.1) -- (\x,\y+0.1);
            }
        }
        
        % Zeichne Elektronen als kleine blaue Kreise mit Pfeilen für Bewegung
        \foreach \pos in {(-0.5,1), (1.5,3), (3.5,1), (5.5,3), (3,2)} {
            \filldraw[black] \pos circle (0.15);
        }
        \draw[black,->] (-1,1) node[left]{$e^-$} -- (1,1);
        \draw[black,->] (1,3) -- (3,3);
        \draw[black,->] (3,1) -- (5,1);
        \draw[black,->] (5,3) -- (7,3) node[right]{$e^-$};
        
        % Stöße der Elektronen mit den Ionenrümpfe
        \draw[black,->] (2.5,2) -- (3.6,2.1) to[out=60,in=200] (4.6,3.5);
        
    \end{tikzpicture}
    \caption{Elektronengas in einem Leiter}
    \label{fig:Elektronengas}
\end{figure}
Legt man eine Spannung an bewegen sich die Elektrone von minus Pol zum plus Pol 
wobei sie mit den Ionenrümpfe zusammenstoßen, wie in Abbildung \ref{fig:Elektronengas} dargestellt.
Durch diese Stöße werden die Elektronen gestreut was den Stromfluss behindert und so zu einem Wiederstand führt. Die Gitter Schwingungen 
hängen von der Temperatur ab, desdo höher die Temperatur, desdo stärker die Schwingungen und desdo mehr Stöße gibt es, was zu einem höheren
Widerstand führt. Umgekehrt ist das natürlich 
auch der Fall. \\

Bei bestimmten Metallen, wie z.B. Quecksilber oder Blei, kann man aber beobachten das der Wiederstand ab einer bestimmten 
Temperatur plötzlich ganz verschwindet. Dieses Phänomen kann durch die BCS-Theorie erklärt werden.
\newpage

\section{BCS-Theorie}
Die BCS-Theorie basiert auf der Annahme, dass Elektronen sich zu sogenannten Cooper-Paaren verbinden. 
Druch die Starke abkühlung des Superleiters sind die Gitterschwingungen im Metallgitter durch die Temperatur 
vernachlässigbar klein. Wenn jetzt ein Elektron durch das Metallgitter fliegt, zieht es die positiv geladenen
Ionenrümpfe an, wodurch um dem Berich um des Elektrons eine lokale Polarisation des Gitters entsthet. Diese 
Gitterpolarisation kann ein weiteres Elektron anziehen\footnote{Diese Anziehung kann zwischen Elektronen statfinden, die im Kristallgitter von bis zu 100 Atomen entfrent sind.}. 

\begin{figure}[!ht]
    \centering
    \begin{tikzpicture}
        \foreach \x in {0,...,7} {
            \foreach \y in {0,...,4} {
                \node[circle, draw=gray!80, fill=gray!40, minimum size=2.5mm] at (\x,\y) {};
                \draw[line width=0.2mm] (\x-0.1,\y) -- (\x+0.1,\y);
                \draw[line width=0.2mm] (\x,\y-0.1) -- (\x,\y+0.1);
            }
        }
        % Elektron 1
        \filldraw[black] (1.5,1.5) circle (0.1);
        \draw [->] (1.12,1.12) -- (1.4,1.4);
        \draw [->] (1.88,1.88) -- (1.6,1.6);
        \draw [->] (1.12,1.88) -- (1.4,1.6);
        \draw [->] (1.88,1.12) -- (1.6,1.4);
        \draw [dashed] (1.5,1.5) circle [radius=1];

        % Elektron 2
        \filldraw[black] (5.5,2.5) circle (0.1);  
        \draw [->] (5.12,2.12) -- (5.4,2.4);
        \draw [->] (5.88,2.88) -- (5.6,2.6);
        \draw [->] (5.12,2.88) -- (5.4,2.6);
        \draw [->] (5.88,2.12) -- (5.6,2.4);
        \draw [dashed] (5.5,2.5) circle [radius=1];
        
        
        \draw [double distance=4, -{Stealth[inset=0, width=10,length=10]}] (2.52,1.75) -- (3.5,2);
        \draw [double distance=4, -{Stealth[inset=0, width=10,length=10]}] (4.52,2.255) -- (3.5,2);
        


       
    \end{tikzpicture}
    \caption{Cooper-Paar Bildung}
    \label{fig:CooperPaar}
\end{figure}
Wenn der Spin $\overrightarrow{s}$\footnote{Bei Elektronen ist der Spin entweder $+1/2$ oder $-1/2$ (UP oder DOWN)} und der Impuls $\overrightarrow{P}$ der beiden Elektronen entgegengesetzt sind, und der 
Abstand zwischen den Elektronen klein genug ist, kann die Coulombabstoßung\footnote{Die Coulombabstoßung ist die Abstoßung zweier Teilchen mit gleicher Ladung, hier die Elektronen.} übertroffen werden. So ein Elektronenpaar nennt man
Cooper-Paar. 

\subsection{Cooper-Paar}
Woher kommt aber nun die Superleitung? \\
Dadurch das Cooper-Paare aus zwei Elektronen bestehen, deren Spin antiparallel ist ist der Gesamtspin des Cooper-Paares $S=0$.
Somit gehört das Cooper-Paar zu den \textit{Bosonen} und nicht mehr zu den \textit{Fermionen}, wozu die Elektronen uhrsprünglich gehörten.
Bosonen unterliegen nicht dem \textit{Pauli-Prinzip}, welches besagt, dass zwei Fermionen nicht den gleichen Quantenzustand haben können.
Die Cooper-Paare können sich also im gleichen Quantenzustand aufhalten, wodurch sie nicht mehr mit dem Metallgitter wechselwirken.
Warum die Cooper-Paare nicht mit dem Metallgitter wechselwirken lässt sich nur noch mithilfe der Quantenmechanik erklären. \\
\\












\section{Anwendungen der Supraleitung}
Supraleitung hat viele potenzielle Anwendungen, darunter:
\begin{itemize}
    \item Verlustfreie Energieübertragung: Da supraleitende Materialien keinen elektrischen Widerstand haben, 
    könnten sie verwendet werden, um elektrische Energie ohne Verluste zu übertragen.
    \item Magnetresonanztomographie (MRT): Supraleitende Magnete werden in \\ MRT-Geräten verwendet, um starke 
    Magnetfelder zu erzeugen.
    \item Teilchenbeschleuniger: Supraleitende Magnete werden auch in Teilchenbeschleunigern wie dem Large 
    Hadron Collider (LHC) verwendet, um Teilchen auf hohe Geschwindigkeiten zu beschleunigen.
\end{itemize}


\begin{tikzpicture}
    % Gitterstruktur mit positiven Atomrümpfen
    \foreach \x in {0,...,6} {
        \foreach \y in {0,...,4} {
            \node[circle, draw=gray!80, fill=gray!40] at (\x,\y) {};
        }
    }

    % Elektron-Gitter-Wechselwirkung (Polaron)
    % Kreise zur Hervorhebung
    \draw[dashed] (2.5,2) circle [radius=1];
    \draw[dashed] (4.5,2) circle [radius=1];

    % Elektronenbewegungspfeile
    \draw[->] (2.5,2) -- (4.5,2);

    % Beschriftungen
    \node[align=center] at (2.5,-0.5) {\textbf{Elektron - Gitter Wechselwirkung}\\(Polaron)};
    
    % Positive Atomrümpfe beschriften
    \node[align=center] at (6,-0.5) {\textbf{positiv geladene Atomrümpfe}};
\end{tikzpicture}
\end{document}