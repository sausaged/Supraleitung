\documentclass[a4paper]{scrartcl}

\usepackage[utf8]{inputenc}
\usepackage[ngerman]{babel}
\usepackage[T1]{fontenc}

\usepackage{hyperref}
\usepackage{graphicx}

\usepackage{tikz}
\usetikzlibrary{decorations.pathreplacing, arrows.meta}

\usepackage{amsfonts}
\usepackage{amsmath}

\title{Supraleitung}
\author{Jonas Leggewie}
\date{\today}

\begin{document}
\maketitle
\newpage
\tableofcontents
\newpage


\section{Einleitung}
Supraleitung beschreibt das Phänomen, bei dem elektrischer Strom, bei sehr niedrigen Temperaturen 
ohne Wiederstand fließt, wodurch eine verlustfreie Energieübertragung möglich Wird. Was die 
Übertragungsverluste auf langen Stromtrassen deutlich reduzieren würde und eine Effizientere 
Energieversorgung ermöglichen könnte. Auserdem haben Supraleiter noch andere spannende 
Eigenschaften, welche noch nicht vollständig verstanden wurden. 


\section{Stromleitung auf der Atomarer Ebene}
Strom fließt in einen Leiter, z.B ein Kupferdraht, indem sich Elektronen durch das Leitermaterial
bewegen. Elektrische Leiter bestehen aus positive geladenen Ionenrümpfe\footnote{Ionenrümpfe sind die Atome des Kristallgitters, die eines oder mehrere ihrer außen Elektronen abgegeben haben und deswegen positiv geladen sind}, die in einem Kristallgitter angeordnet sind. Um 
die Ionenrümpfe befinden sich die Elektronen welche wie in einem Gas frei um die Ionenrümpfe herum fliegen.


\begin{figure}[h]
    
    \centering
    \begin{tikzpicture}
        % Zeichne die Ionenrümpfe als Kreise im Kristallgitter
        \foreach \x in {0,2,4,6} {
            \foreach \y in {0,2,4} {
                \node[circle, draw=gray!80, fill=gray!40, minimum size=7.5mm] at (\x,\y) {};
                \draw[line width=0.5mm] (\x-0.1,\y) -- (\x+0.1,\y);
                \draw[line width=0.5mm] (\x,\y-0.1) -- (\x,\y+0.1);
            }
        }
        
        % Zeichne Elektronen als kleine blaue Kreise mit Pfeilen für Bewegung
        \foreach \pos in {(-0.5,1), (1.5,3), (3.5,1), (5.5,3), (3,2)} {
            \filldraw[black] \pos circle (0.15);
        }
        \draw[black,->] (-1,1) node[left]{$e^-$} -- (1,1);
        \draw[black,->] (1,3) -- (3,3);
        \draw[black,->] (3,1) -- (5,1);
        \draw[black,->] (5,3) -- (7,3) node[right]{$e^-$};
        
        % Stöße der Elektronen mit den Ionenrümpfe
        \draw[black,->] (2.5,2) -- (3.6,2.1) to[out=60,in=200] (4.6,3.5);
        
    \end{tikzpicture}
    \caption{Elektronengas in einem Leiter}
    \label{fig:Elektronengas}
\end{figure}
Legt man eine Spannung an bewegen sich die Elektrone von minus Pol zum plus Pol 
wobei sie mit den Ionenrümpfe zusammenstoßen, wie in Abbildung \ref{fig:Elektronengas} dargestellt.
Durch diese Stöße werden die Elektronen gestreut was den Stromfluss behindert und so zu einem Wiederstand führt. Die Gitter Schwingungen 
hängen von der Temperatur ab, desdo höher die Temperatur, desdo stärker die Schwingungen und desdo mehr Stöße gibt es, was zu einem höheren
Widerstand führt. Umgekehrt ist das natürlich 
auch der Fall. \\

Bei bestimmten Metallen, wie z.B. Quecksilber oder Blei, kann man aber beobachten das der Wiederstand ab einer bestimmten 
Temperatur plötzlich ganz verschwindet. Dieses Phänomen kann durch die BCS-Theorie erklärt werden.
\newpage

\section{BCS-Theorie}
Die BCS-Theorie basiert auf der Annahme, dass Elektronen sich zu sogenannten Cooper-Paaren verbinden. 
Druch die Starke abkühlung des Superleiters sind die Gitterschwingungen im Metallgitter durch die Temperatur 
vernachlässigbar klein. Wenn jetzt ein Elektron durch das Metallgitter fliegt, zieht es die positiv geladenen
Ionenrümpfe an, wodurch um dem Berich um des Elektrons eine lokale Polarisation des Gitters entsthet. Diese 
Gitterpolarisation kann ein weiteres Elektron anziehen\footnote{Diese Anziehung kann zwischen Elektronen statfinden, die im Kristallgitter von bis zu 100 Atomen entfrent sind.}. 

\begin{figure}[!ht]
    \centering
    \begin{tikzpicture}
        \foreach \x in {0,...,7} {
            \foreach \y in {0,...,4} {
                \node[circle, draw=gray!80, fill=gray!40, minimum size=2.5mm] at (\x,\y) {};
                \draw[line width=0.2mm] (\x-0.1,\y) -- (\x+0.1,\y);
                \draw[line width=0.2mm] (\x,\y-0.1) -- (\x,\y+0.1);
            }
        }
        % Elektron 1
        \filldraw[black] (1.5,1.5) circle (0.1);
        \draw [->] (1.12,1.12) -- (1.4,1.4);
        \draw [->] (1.88,1.88) -- (1.6,1.6);
        \draw [->] (1.12,1.88) -- (1.4,1.6);
        \draw [->] (1.88,1.12) -- (1.6,1.4);
        \draw [dashed] (1.5,1.5) circle [radius=1];

        % Elektron 2
        \filldraw[black] (5.5,2.5) circle (0.1);  
        \draw [->] (5.12,2.12) -- (5.4,2.4);
        \draw [->] (5.88,2.88) -- (5.6,2.6);
        \draw [->] (5.12,2.88) -- (5.4,2.6);
        \draw [->] (5.88,2.12) -- (5.6,2.4);
        \draw [dashed] (5.5,2.5) circle [radius=1];
        
        
        \draw [double distance=4, -{Stealth[inset=0, width=10,length=10]}] (2.52,1.75) -- (3.5,2);
        \draw [double distance=4, -{Stealth[inset=0, width=10,length=10]}] (4.52,2.255) -- (3.5,2);
        


       
    \end{tikzpicture}
    \caption{Cooper-Paar Bildung}
    \label{fig:CooperPaar}
\end{figure}
Wenn der Spin $\overrightarrow{s}$\footnote{Bei Elektronen ist der Spin entweder $+1/2$ oder $-1/2$ (UP oder DOWN)} und der Impuls $\overrightarrow{P}$ der beiden Elektronen entgegengesetzt sind, und der 
Abstand zwischen den Elektronen klein genug ist, kann die Coulombabstoßung\footnote{Die Coulombabstoßung ist die Abstoßung zweier Teilchen mit gleicher Ladung, hier die Elektronen.} übertroffen werden. So ein Elektronenpaar nennt man
Cooper-Paar. 

\subsection{Cooper-Paar}
Woher kommt aber nun die Superleitung? \\
Dadurch das Cooper-Paare aus zwei Elektronen bestehen, deren Spin antiparallel ist ist der Gesamtspin des Cooper-Paares $S=0$.
Somit gehört das Cooper-Paar zu den \textit{Bosonen} und nicht mehr zu den \textit{Fermionen}, wozu die Elektronen uhrsprünglich gehörten.
Bosonen unterliegen nicht dem \textit{Pauli-Prinzip}, welches besagt, dass zwei Fermionen nicht den gleichen Quantenzustand haben können.
Die Cooper-Paare können sich also im gleichen Quantenzustand aufhalten, wodurch sie nicht mehr mit dem Metallgitter wechselwirken.
Warum die Cooper-Paare nicht mit dem Metallgitter wechselwirken lässt sich nur noch mithilfe der Quantenmechanik erklären. \\
\\


\section{Meißner-Ochsenfeld-Effekt}
Ein weiteres Phänomen der Supraleitung ist der Meißner-Ochsenfeld-Effekt. FRITZ WALTER MEIßNER und 
ROBERT OCHSENFELD entdeckten 1933, dass Magnetfelder aus dem Inneren eines Supraleiters verdrängt werden. 
Dieser Effekt tritt auf, wenn ein Supraleiter unter seine kritische Temperatur abgekühlt wird. Somit sind 
Supraleiter nicht nur ideale Leiter, sondern auch perfekte Diamagneten.
\begin{figure}[!h]
    \centering
    \begin{tikzpicture}
        % Left diagram: T > Tc
        \begin{scope}[xshift=-4cm]
            
            
            % Magnetic field lines (straight)
            \foreach \y in {-2, -1.5, -1, -0.5, 0, 0.5, 1, 1.5, 2} {
                \draw[thick, ->] (-3,\y) -- (3,\y);
            }
            % Ball
            \shade[ball color=gray, opacity=0.9] (0,0) circle (1);
            
            % Label
            \node at (0,-2.8) {\(T > T_c\)};
        \end{scope}
        
        % Right diagram: T < Tc
        \begin{scope}[xshift=4cm]
            % Streamlines that curve around the ball
            \draw [thick, ->] (-3, -2) -- (3,-2);
            \draw [thick, ->] (-3, -1.5) -- (-1.5,-1.5) to[out=-20, in=-160] (1.5,-1.5) -- (3,-1.5);
            \draw [thick, ->] (-3, -1) -- (-1.5,-1) to[out=-30, in=-150] (1.5,-1) -- (3,-1);
            \draw [thick, ->] (-3, -0.5) -- (-1.5,-0.5) to[out=-50, in=-130] (1.5,-0.5) -- (3,-0.5);
            %\draw [thick, ->] (-3, 0) -- (3,0);
            \draw [thick, ->] (-3, 0.5) -- (-1.5,0.5) to[out=50, in=130] (1.5,0.5) -- (3,0.5);
            \draw [thick, ->] (-3, 1) -- (-1.5,1) to[out=30, in=150] (1.5,1) -- (3,1);
            \draw [thick, ->] (-3, 1.5) -- (-1.5,1.5) to[out=20, in=160] (1.5,1.5) -- (3,1.5);
            \draw [thick, ->] (-3, 2) -- (3,2);
            
            % Ball
            \shade[ball color=gray, opacity=0.9] (0,0) circle (1);
            
            % Label
            \node at (0,-2.8) {\(T < T_c\)};
        \end{scope}
    \end{tikzpicture}
    \caption{Meißner-Ochsenfeld-Effekt}
\end{figure}

Nimmt man ein Superleitendes Material mit einer Temperatur höher als die kritische Temperatur $T_c$
und setzt es einem Magnetfeld aus, so dringt das Magnetfeld nahezu ungehindert durch das Material (wie links in Abbildung 3 zusehen ist). 
Wenn man den Supraleiter jetzt unter die kritische Temperatur abkühlt, wird das Magnetfeld aus dem Inneren des Supraleiters verdrängt (Abbildung 3 rechts).
Doch eigentlich müsste doch das Magnetfeld weiterhin ungehindert das Material durchdringen können, da der 
Widerstand gleich null ist, kann keine Spannung abfallen oder induziert werden, wodurch sich das Magnetfeld 
eigentlich nicht verändern dürfte.

Die Erklärung für dieses Phänomen ist die Londonsche Eindringtiefe. FRITZ und HANS LONDON
versuchten 1935 die charakteristischen Eigenschaften der Supraleitung durch ihre London-Gleichungen 
zu beschreiben. Daraus folgte dann das ein äußeres Magnetfeld doch in eine dünne Oberflächenschicht des Supraleiters eindringt (Eindringtiefe $\lambda_L$).

\begin{equation}
    \vec{B}(x) = \vec{B}_0 e^{-\frac{x}{\lambda_L}}
\end{equation}

Dabei ist $\vec{B}$ das Magnetfeld im Supraleiter, $\vec{B}_0$ das Magnetfeld außerhalb des Supraleiters, $d$ die Tiefe in den Supraleiter und $\lambda_L$ die Londonsche Eindringtiefe.
\\
\\
Das Magnetfeld, welches in diese dünne Schicht eindringt, führt zu einem supraleitendem Stromfluss, welcher ein
Magnetfeld erzeugt, was gerade so stark ist um das äußere Magnetfeld aufzuheben, was 
die Begründung für den Meißner-Ochsenfeld-Effekt ist. 



\section{Anwendungen der Supraleitung}
Die wohl theoretisch simpelste Anwendung eines Supraleiters wäre, ihn als Ersatz für klassische Stromleitungen zu 
verwenden. So konnte man zum Beispiel die Längste Deutsche Stromtrasse(SüdLink), mit einer Länge von $700km$, 
durch Supraleitung ersetzen. 

Um den die Verlustleistung $P_V$ auf z. B. der SüdLink Stromtrasse zu berechnen kann man folgende Formel verwenden:
\begin{equation}
    P_V = R \cdot I^2
\end{equation}
wobei $R$ der Widerstand der Leitung in Ohm und $I$ der Strom in Ampere ist.
Die SüdLink Stromtrasse wurde als Hochspannungs-Gleichstrom-Übertragung(HGÜ) konzipiert, was die Verlustleistung
deutlich reduziert, da hier der Strom mit hoher Spannung und niedriger Stromstärke übertragen wird und die Verlustleistung
quadratisch mit der Stromstärke steigt.

Um die Formel anwenden zu können brachen wir noch den Widerstand der Leitung. Der Widerstand einer Leitung lässt sich
mit der Formel:
\begin{equation}
    R = \rho \cdot \frac{L}{A}
\end{equation}
berechnen, wobei $\rho$ der spezifische Widerstand des Materials in Ohm Meter, $L$ die Länge der Leitung in Meter und $A$
die Querschnittsfläche der Leitung in Quadratmeter ist. Für Kupfer beträgt der spezifische Widerstand $\rho = 1.68 \cdot 10^{-8} \Omega \cdot m$.
Der Querschnitt der SüdLink Stromtrasse beträgt $A=2100mm^2 = 0,0021m^2$. Daraus folgt:
\begin{equation}
    R = 1.68 \cdot 10^{-8} \Omega \cdot m \cdot \frac{700000m}{0.0021m^2} = 5.6 \Omega
\end{equation}
Nun brauchen wir nur noch den Strom $I$ um die Verlustleistung zu berechnen. Eine typische Spannung für eine HGÜ Trasse 
ist $U = 525kV = 525000V$. SüdLing hat eine Übertragungsleistung von $P = 4GW = 4000000000W$. So ergibt sich der Strom mit folgender Formel:
\begin{equation}
    I = \frac{P}{U} = \frac{4 \cdot 10^9W}{525000V} \approx 7619.05A
\end{equation}
Jetzt können wir die Werte in die Formel für die Verlustleistung einsetzen:
\begin{equation}
    P_V = 5.6 \Omega \cdot (7619.05A)^2 \approx 3.25 \cdot 10^8W = 325MW
\end{equation}

Also beträgt der Leistungsverlust $P_V = 235MW$. Diese Berechnung ist aber nur eine grobe Schätzung, da die Stromtrasse natürlich 
nicht aus einen kontinuierlichem 700 km langen Kupferdraht besteht. Es gibt deshalb noch die Faustregel, dass die Leistung auf 
100 km um 0,5\% sinkt. Also bei 700 km wären das 3,5\%. 3,5\% von 4GW sind 140MW was der Leistungsverlust auf der SüdLink Stromtrasse
nach der Faustregel wäre. Aber 140MW sind immer noch eine Menge Energie die verloren geht. Mit 140MW könnte man zum Beispiel
den durchschnittlichen Strombedarf von 70.000 bis 140.000 Haushalten decken, was ungefähr der Größe einer Stadt wie Bonn.
Wenn die Stromtrasse aus Supraleitern bestehen würde, wäre der Leistungsverlust gleich null, da Supraleiter keinen Widerstand haben.
Allerdings sind 3,5\% Leistungsverlust auch nicht sehr viel. Außerdem müssen Supraleiter auch ständig gekühlt werden, weshalb sie 
wirtschaftlich nicht um bedingt besser sind. Allerdings finden Supraleiter schon heute Anwendung im kleineren Maßstab. In
Städten können diese gut Platz sparen, da ein Supraleiter eine viel geringere Spannung als ein normaler Leiter benötigt.
So kann ein Umspannwerk für 110kV, welches der Fläche einer Turnhalle entspricht auf ein Umspannwerk für 10kV, welches nur die
Fläche einer Doppelgarage benötigt, ersetzt werden. 

Auch in der medizinischen Magnetresonanztomografie (MRT) kommen Supraleiter in einem bedeutenden Anwendungsbereich zum Einsatz. Supraleiter kommen hier zum Einsatz, um die für die Bildgebung erforderlichen extrem starken Magnetfelder zu erzeugen. Da Supraleiter keinen elektrischen Widerstand besitzen, sind diese Magnetfelder stabil und effizient. 
Supraleiter werden auch in der Teilchenphysik eingesetzt, etwa in Teilchenbeschleunigern wie dem Large Hadron Collider (LHC). Supraleitende Magnete kommen dort zum Einsatz, um die Teilchenstrahlen auf ihrer Bahn zu halten und zu bündeln.
Supraleitende Quantencomputer stellen ein zusätzliches Anwendungsfeld dar. Mit supraleitenden Schaltkreisen können Qubits realisiert werden, die für Quantenberechnungen essenziell sind. 

Zudem finden Supraleiter in der Energietechnik Anwendung, wie etwa in supraleitenden Stromspeichern (SMES), die elektrische Energie verlustfrei speichern. In der Zukunft könnte diese Technologie eine bedeutende Funktion bei der Stabilisierung von Stromnetzen übernehmen. 
Es existieren auch Anwendungen in der Magnetbahntechnologie, wie bei Magnetschwebebahnen (Maglev). Supraleiter machen hier die Bildung intensiver Magnetfelder möglich, die das Schwebebetriebs- und Reibungslosigkeitsprinzip der Züge gewährleisten. 


\end{document}