\subsection{Grenzen der Supraleitung}
Die Supraleitung wird durch drei kritische Parameter begrenzt:

\begin{itemize}
    \item \textbf{Kritische Temperatur} ($T_c$): Oberhalb dieser Temperatur geht der supraleitende Zustand verloren. Dies liegt daran, dass die thermische Energie der Elektronen die Bildung von Cooper-Paaren verhindert, die für die Supraleitung verantwortlich sind.
    \item \textbf{Kritische Feldstärke} ($H_c$): Bei Überschreiten dieser magnetischen Feldstärke wird die Supraleitung zerstört. Ein zu starkes Magnetfeld kann die Cooper-Paare aufbrechen und die supraleitende Phase in einen normalen Zustand überführen.
    \item \textbf{Kritische Stromdichte} ($J_c$): Wird die maximale Stromdichte überschritten, bricht der supraleitende Zustand zusammen. Dies geschieht, weil die durch den Strom erzeugte Lorentzkraft die Cooper-Paare destabilisiert und die Supraleitung unterbricht.
\end{itemize}
