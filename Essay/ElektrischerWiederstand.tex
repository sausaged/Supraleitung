Der elektrische Widerstand ist eine Eigenschaft von Materialien, welche den Fluss von elektrischen
Ladungen behindert. Er wird in Ohm gemessen und ist abhängig von der Temperatur und dem Material.
Der Widerstand eines Leiters ist gegeben durch das Ohmsche Gesetz:
\begin{equation}
    R = \frac{U}{I}
\end{equation}
wobei $R$ der Widerstand, $U$ die Spannung und $I$ der Strom ist.


\subsection{Temperaturabhängigkeit des Widerstands}
Wo steckt nun die Temperaturabhängigkeit?

Der Widerstand eines bestimmten Leiterabschnitts kann durch die Formel
\begin{equation}
    R = \rho \frac{L}{A}
\end{equation}
beschrieben werden, wobei $\rho$ der spezifische Widerstand des Materials, $L$ die Länge des Leiters und $A$ die Querschnittsfläche des Leiters ist. Der spezifische Widerstand $\rho$ ist eine Materialkonstante und hängt stark von der Temperatur ab. Bei den meisten Materialien nimmt der spezifische Widerstand mit steigender Temperatur zu, was bedeutet, dass der Widerstand des Leiters ebenfalls zunimmt.

Für Metalle kann der Zusammenhang zwischen dem spezifischen Widerstand und der Temperatur oft näherungsweise durch die Formel
\begin{equation}
    \rho(T) = \rho_0 (1 + \alpha (T - T_0))
\end{equation}
beschrieben werden, wobei $\rho_0$ der spezifische Widerstand bei einer Referenztemperatur $T_0$ und $\alpha$ der Temperaturkoeffizient des Widerstands ist. Diese Beziehung zeigt, dass der spezifische Widerstand und damit der Widerstand des Leiters mit steigender Temperatur linear zunimmt.
