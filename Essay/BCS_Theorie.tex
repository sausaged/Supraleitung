\section{BCS-Theorie}
Die BCS-Theorie basiert auf der Annahme, dass Elektronen sich zu sogenannten Cooper-Paaren verbinden. 
Druch die Starke abkühlung des Superleiters sind die Gitterschwingungen im Metallgitter durch die Temperatur 
vernachlässigbar klein. Wenn jetzt ein Elektron durch das Metallgitter fliegt, zieht es die positiv geladenen
Ionenrümpfe an, wodurch um dem Berich um des Elektrons eine lokale Polarisation des Gitters entsthet. Diese 
Gitterpolarisation kann ein weiteres Elektron anziehen\footnote{Diese Anziehung kann zwischen Elektronen statfinden, die im Kristallgitter von bis zu 100 Atomen entfrent sind.}. 

\begin{figure}[!ht]
    \centering
    \begin{tikzpicture}
        \foreach \x in {0,...,7} {
            \foreach \y in {0,...,4} {
                \node[circle, draw=gray!80, fill=gray!40, minimum size=2.5mm] at (\x,\y) {};
                \draw[line width=0.2mm] (\x-0.1,\y) -- (\x+0.1,\y);
                \draw[line width=0.2mm] (\x,\y-0.1) -- (\x,\y+0.1);
            }
        }
        % Elektron 1
        \filldraw[black] (1.5,1.5) circle (0.1);
        \draw [->] (1.12,1.12) -- (1.4,1.4);
        \draw [->] (1.88,1.88) -- (1.6,1.6);
        \draw [->] (1.12,1.88) -- (1.4,1.6);
        \draw [->] (1.88,1.12) -- (1.6,1.4);
        \draw [dashed] (1.5,1.5) circle [radius=1];

        % Elektron 2
        \filldraw[black] (5.5,2.5) circle (0.1);  
        \draw [->] (5.12,2.12) -- (5.4,2.4);
        \draw [->] (5.88,2.88) -- (5.6,2.6);
        \draw [->] (5.12,2.88) -- (5.4,2.6);
        \draw [->] (5.88,2.12) -- (5.6,2.4);
        \draw [dashed] (5.5,2.5) circle [radius=1];
        
        
        \draw [double distance=4, -{Stealth[inset=0, width=10,length=10]}] (2.52,1.75) -- (3.5,2);
        \draw [double distance=4, -{Stealth[inset=0, width=10,length=10]}] (4.52,2.255) -- (3.5,2);
        


       %2.5,2
    \end{tikzpicture}
    \caption{Cooper-Paar Bildung}
    \label{fig:CooperPaar}
\end{figure}
Wenn der Spin $\overrightarrow{s}$ und der Impuls $\overrightarrow{P}$ der beiden Elektronen entgegengesetzt sind, und der 
Abstand zwischen den Elektronen klein genug ist, kann die Coulombabstoßung übertroffen werden. So ein Elektronenpaar nennt man
Cooper-Paar. 

